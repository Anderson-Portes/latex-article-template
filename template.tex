\documentclass[12pt,a4paper]{article}
\usepackage{lipsum}
\usepackage{indentfirst}
\usepackage{setspace}
\usepackage{ragged2e}
\usepackage{setspace}
\usepackage[utf8]{inputenc}
\usepackage[english,portuguese]{babel}
\usepackage[lmargin=3cm,tmargin=3cm,rmargin=2cm,bmargin=2cm]{geometry}

\newenvironment{long-quote}{\begin{quotation}\singlespacing\small\noindent}{\end{quotation}}

\setlength{\parindent}{1.25cm}
\onehalfspacing
\setlength{\parskip}{6pt} 

\begin{document}

    \frenchspacing

    \title{
        Título do artigo \footnote{Obetivo do trabalho ...}
    }
    \author{
        Anderson Portes do Nascimento \footnote{Estudante da faculdade ...}
    }
    \date{2023, v1.0}
    \maketitle

    \begin{abstract}
        \lipsum[1]

        \bigskip

        \noindent\textbf{Palavras-Chaves}: Palavra1, Palavra2, Palavra3.
    \end{abstract}

    \begin{otherlanguage}{english}
        \begin{abstract}
            \lipsum[1]

            \bigskip

            \noindent\textbf{Keywords}: Word1, Word2, Word3.
        \end{abstract}
    \end{otherlanguage}
    
    \section{Introdução}
        \lipsum[1]

        \lipsum[1]
    \section{Procedimentos Teórico-Metodológicos}
        "A random text to make a quote example" (EXAMPLE; 2023)
    \section{Fundamentação}
        \lipsum[1]
        
        \begin{long-quote}
            "A citação longa é um trecho de texto com mais de três linhas, com recuo de 4 cm em relação à margem esquerda, fonte menor e espaçamento simples entre linhas" (Sobrenome, Ano, p. 10).
        \end{long-quote}

        \lipsum[1]
    \section{Resultados}
        \lipsum[1]
    \section{Considerações Finais}
        \lipsum[1]
    \section{Referências}
\end{document}